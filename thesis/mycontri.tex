% ------------------------------------------------------------------------
% -*-TeX-*- -*-Hard-*- Smart Wrapping
% ------------------------------------------------------------------------
% Thesis Acknowledgements ------------------------------------------------

\prefacesection{My Contribution}
\def\baselinestretch{0.5}
\setlinespacing{1.25}


One of the characteristics of the experiments in the field of the High Energy
Physics is that they are based on the teamwork. In the CMS experiment a few thousands of
scientists, engineers and technicians were involved. Hence, it is obvious that not all issues
discussed in this thesis are my exclusive contribution.
The GEM detector trigger and DAQ sytem was proposed and designed by the Université libre de Bruxelles CMS
group, the group developed most of the custom electronic boards of the trigger system,
prepared the firmware for the FPGA devices, carried out the production and installation of the
trigger electronics. The GEM chambers were developed and produced by the scientists from
Italy, USA, Pakistan, Beljium, India and CERN. The Brussel CMS group have consisted
of a few dozens of people from the de de l'Université libre de Bruxelles. I have been a member of the group for over three years. My tasks included: software development, testing the prototypes of the electronic boards, testing the GEM detectors and DAQ system during installation, proposing the firmware improvements and modifications, work on the optohybird modules.
The on-line software for the PAC Trigger system, that is the subject of the
Chapter 4, was developed mainly by two people: Michał Pietrusiński and me. Michał was themain architect of the software structure and implemented most of the low-level software.
Based on that part, I have designed and implemented the test procedures for the trigger system
as well as details of the hardware configuration process. Additionally, my task was to decide
what diagnostic and monitoring tools should be implemented in the firmware of the trigger
electronics and how to analyse and present the data acquired by those tools. The dedicated
monitoring procedures were developed mostly by me and are a part of the PACT system
online software. 
The hardware and firmware solutions for the synchronization of the chamber data
and transmission channels were created by the main developer of the firmware for the PAC
system – Yifan young. My task was to find the ways of using those solutions in
practice. I have worked out the methods for finding the optimal values of the synchronization
parameters and implemented them in the dedicated software procedures, which allowed
successful synchronization of the VFAT signal to optohybrid module. (at the moment for the cosmic muons). The
analysis of the system synchronization from the data acquired during the cosmic muon runs,
as well as the simulation of the muon hits timing, was performed by other members of the
Brussel CMS group.
