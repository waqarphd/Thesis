% ************************** Thesis Abstract *****************************
% Use `abstract' as an option in the document class to print only the titlepage and the abstract.
\begin{abstract}
The developments were rather generic, driven by the observation that modern technologies allow to design a DAQ architecture independent of the detector technology to which the DAQ system will be connected, providing freedom to the choice of the future experiment. It becomes clear now that the future particle and astro-particle experiments plan to use the most advanced technologies from the telecommunication and the digital programmable electronic industries: the Advanced Telecom Computing Architecture (ATCA or micro-TCA) standard and Field Programmable Gate Arrays (FPGA).Currently we are working on design of the DAQ system of the CMS Forward Muon Upgrade project which proposes to install Triple-GEM detectors instead of Resistive Plate Chambers in the first CMS muon station at 1.6 <$\eta$< 2.4 during the 2nd LHC long shutdown. The IIHE is also leading the design of the DAQ system of Askaryan Radiotelescope Array (ARA) project, for which radio-antennas are being spread over an area of several km2 in the South Pole ice, close to the IceCube experiment. In addition the IIHE contributes to the DAQ development of the Large TPC prototype for the ILC. In the framework of the CMS upgrades, the IIHE in collaboration with the University of Antwerpen is investigating the new micro-TCA standard, introduced by the telecommunication industry, to replace the VME electronics. With its high data throughput (several Gbps), high reliability and high availability, the micro-TCA standard combined with the most powerful FPGAs for the data processing should allow the future CMS DAQ system to cope with the LHC luminosity beyond the nominal value of 10$^{33}$ cm$^{-2}$s$^{-1}$.
Typically these systems require high data volume transmission, from 0.5 to 3.2 Gbps, from the detector to off-detector electronics where data are processed by FPGAs, as well as the distribution of precise signals like clock and trigger inputs towards the detectors. The high bandwidth transmission uses optical fibre transceivers and gigabit transceiver blocks (GTPs) routinely built into FPGA devices. We have also set-up a micro-TCA test bench equipped with various Advance Mezzanine Cards (AMC), either commercial ones or designed by IIHE electronic engineers. In particular we are evaluating the CERN’s Gigabit Link interface Board (GLIB) designed to test the GBT/Versatile optical link being developed by CERN for the next Tracker upgrade.

The micro-TCA technology is a relatively new telecommunication standard. It is planned to replace VME as the default technology for future particle physics experiments (LHC upgrades, ILC experiments, etc.). Consequently it still needs a lot of developments and specification validation for particle physics. Another aspect we are starting to develop is the implementation of complex algorithms on FPGAs. Indeed with the introduction of increasingly powerful FPGAs, it seems now feasible to run on such devices more advanced algorithms for trigger and track reconstruction.  These developments are of interest for the LHC trigger upgrades and in particular for the future CMS track trigger.
 
\end{abstract}
